\chapter{Dark matter}

% esempio di fluidodinamica
% quasi tutta la materia è così, battuta sui fisici
% la macchia che immagina i mondi paralleli
% spiegare che la forma è circa tonda quindi è improbabile, attrito,
%   momento angolare
% in realtà non ero sicuro, citare tulin2017 figura pag 21,
%   e le galassie ellittiche sono grandi. la figura e la spiegazione la prendo
%   da tulin2017, però cito l'articolo originale vogelsberger2012. prendere
%   solo i primi due pannelli.
% bullet cluster per l'autointerazione, figura pag 29 tulin, ovvero clowe2006
%   p 3, citare
%   harvey2015 (collisione di cluster) per il limite su sigma/m < 0.84 barn/GeV,
%   da confrontare con la densità locale 1 GeV/cm^3 [buch2019, Gaia], e con il
%   raggio del protone, 1 fm = 0.1 sqrt(barn), meglio con la sezione d'urto del
%   neutrone, 1 barn (massa 1 GeV).
% insomma, la struttura è ricca come mostrato nelle figure, e magari
%   interagisce, tranne che: devo prendere il raggio atomico, sicuramente non
%   sono atomi di idrogeno, che hanno sezione d'urto 10^10 maggiore. bon.
% altro confronto: il vento solare è qualche particella al cm^3 a 1 AU
% altre cose fighe? buchi neri? no, perché me l'ha detto un tizio al kavli

% Evidenze della materia oscura:
% gravitazionali: velocità orbitale nelle stelle, lenti gravitazionali
% cosmologiche: variazione di densità. il rapporto lineare dal disaccoppiamento
%   è 1000, e basta che faccio vedere che in linea di principio si può
%   calcolare. il PDG sec. 27 dice che le oscillazioni della densità vanno
%   giù linearmente con la scala, mentre usando la varianza mi aspetteri
%   a^(3/2). Idea: lo spettro è invariante di scala perché le interazioni
%   sono locali rispetto alle scale che mi interessano. questo forse mi dà
%   un fattore a^(-1/2) se fisso la scala che guardo?
