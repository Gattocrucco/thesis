\chapter{Additional figures for \protect\autoref{ch:anal}}
\label{ch:analplot}

\begin{figure}[h]
    
    \widecenter{\includempl{figbaseline2-0}}
    
    \widecenter{\includempl{figbaseline2-1}}
    
    \widecenter{\includempl{figbaseline2-2}}
    
    \figcaption{baseline2}{Distribution of the waveform baseline measured in
    the pre-trigger region of the events.}
    
\end{figure}

\begin{figure}
    
    \widecenter{\includempl{figbsoutlier2-00}\includempl{figbsoutlier2-01}}
    
    \widecenter{\includempl{figbsoutlier2-10}\includempl{figbsoutlier2-11}}
    
    \widecenter{\includempl{figbsoutlier2-20}\includempl{figbsoutlier2-21}}

    \figcaption{bsoutlier2}{The events with the lowest and highest baseline.}

\end{figure}

\begin{figure}
    
    \widecenter{\includempl{figbstail2-00}\includempl{figbstail2-01}}
    
    \widecenter{\includempl{figbstail2-10}\includempl{figbstail2-11}}
    
    \widecenter{\includempl{figbstail2-20}\includempl{figbstail2-21}}

    \figcaption{bstail2}{A pair of events from the low baseline tail (baseline
    between 955 and 956).}

\end{figure}

\begin{figure}
    
    \widecenter{\includempl{fighist2dtile21-0}}

    \widecenter{\includempl{fighist2dtile21-1}}

    \widecenter{\includempl{fighist2dtile21-2}}
    
    \figcaption{hist2dtile21}{Time-value histograms of LFoundry tile~21 in LNGS
    laser data at overvoltages \SI{5.5}V, \SI{7.5}V, and~\SI{9.5}V.}
    
\end{figure}

\begin{figure}
    
    \widecenter{\includempl{figlaserpos2-00}\includempl{figlaserpos2-01}}

    \widecenter{\includempl{figlaserpos2-10}\includempl{figlaserpos2-11}}

    \widecenter{\includempl{figlaserpos2-20}\includempl{figlaserpos2-21}}

    \figcaption{laserpos2}{In each row, left panel: histogram of the laser peak
    position, only for 1 PE peaks, for all filter lengths. The zero of the
    scale is such that the expected position is 8969. Right panel: 2D histogram
    of the laser peak position and the amplitude, for the same selection of
    peaks, but only with the principal filter length used in the rest of the
    analysis.}
    
\end{figure}

\begin{figure}
    
    \widecenter{\includempl{figlptail2-00}\includempl{figlptail2-01}}

    \widecenter{\includempl{figlptail2-10}\includempl{figlptail2-11}}

    \widecenter{\includempl{figlptail2-20}\includempl{figlptail2-21}}
    
    \figcaption{lptail2}{In each row, left panel: the event with the rightmost
    laser peak position for 1 PE peaks with filter length \SI{64}{ns}. Right
    panel: the same event with filter length \SI{128}{ns}.}
    
\end{figure}

\begin{figure}
    
    \widecenter{\includempl{figverymissing-00}\includempl{figverymissing-01}\includempl{figverymissing-02}}

    \widecenter{\includempl{figverymissing-03}\includempl{figverymissing-04}\includempl{figverymissing-05}}

    \widecenter{\includempl{figverymissing-06}\includempl{figverymissing-07}\includempl{figverymissing-08}}

    \widecenter{\includempl{figverymissing-09}\includempl{figverymissing-010}\includempl{figverymissing-011}}

    \caption{\label{fig:verymissing0} Twelve randomly selected events at
    \SI{5.5}{VoV} where with all filter lengths there is no local minimum in
    the laser peak search range. \scriptlink{figverymissing.py}}

\end{figure}

\begin{figure}
    
    \widecenter{\includempl{figverymissing-10}\includempl{figverymissing-11}\includempl{figverymissing-12}}

    \widecenter{\includempl{figverymissing-13}\includempl{figverymissing-14}\includempl{figverymissing-15}}

    \widecenter{\includempl{figverymissing-16}\includempl{figverymissing-17}\includempl{figverymissing-18}}

    \widecenter{\includempl{figverymissing-19}\includempl{figverymissing-110}\includempl{figverymissing-111}}

    \caption{\label{fig:verymissing1} Twelve randomly selected events at
    \SI{7.5}{VoV} where with all filter lengths there is no local minimum in
    the laser peak search range. \scriptlink{figverymissing.py}}

\end{figure}

\begin{figure}
    
    \widecenter{\includempl{figverymissing-20}\includempl{figverymissing-21}\includempl{figverymissing-22}}

    \widecenter{\includempl{figverymissing-23}\includempl{figverymissing-24}\includempl{figverymissing-25}}

    \widecenter{\includempl{figverymissing-26}\includempl{figverymissing-27}\includempl{figverymissing-28}}

    \widecenter{\includempl{figverymissing-29}\includempl{figverymissing-210}\includempl{figverymissing-211}}

    \caption{\label{fig:verymissing2} Twelve randomly selected events at
    \SI{9.5}{VoV} where with all filter lengths there is no local minimum in
    the laser peak search range. \scriptlink{figverymissing.py}}

\end{figure}

\begin{figure}
    
    \widecenter{\includempl{figpe2-0}}

    \widecenter{\includempl{figpe2-1}}

    \widecenter{\includempl{figpe2-2}}
    
    \figcaption{pe2}{Histogram of the laser peak height, with a selection to
    avoid biased heights. The dotted lines are the boundaries of the PE bins.}
    
\end{figure}

\begin{figure}
    
    \widecenter{\includempl{figpretrigger2-00}\includempl{figpretrigger2-01}}

    \widecenter{\includempl{figpretrigger2-10}\includempl{figpretrigger2-11}}

    \widecenter{\includempl{figpretrigger2-20}\includempl{figpretrigger2-21}}

    \figcaption{pretrigger2}{In each row, left panel: scatter plot amplitude
    vs.\ position of the chronological first pre-trigger peak in each event.
    Right panel: histogram of the amplitude for the peaks contained in the
    vertical gray band in the left plot.}
    
\end{figure}

\begin{figure}
    
    \widecenter{\includempl{figapscatter2-00}\includempl{figapscatter2-01}}

    \widecenter{\includempl{figapscatter2-10}\includempl{figapscatter2-11}}

    \widecenter{\includempl{figapscatter2-20}\includempl{figapscatter2-21}}

    \figcaption{apscatter2}{In each row, left panel: corrected amplitude versus
    delay from laser peak; the gray bands mark the selected ranges. Right
    panel: zoom on the selection corner.}

\end{figure}

\begin{figure}
    
    \widecenter{\includempl{figapfit2-00}\includempl{figapfit2-01}}

    \widecenter{\includempl{figapfit2-10}\includempl{figapfit2-11}}

    \widecenter{\includempl{figapfit2-20}\includempl{figapfit2-21}}

    \figcaption{apfit2}{In each row, left panel: fit of the temporal
    distribution of post-trigger pulses. Right panel: the histogram with even
    bins and without the temporal cuts. In the legend, `const' is the
    background excess density, i.e.\ $(R/N)/((t_R-t_L)(1-R/N))$, in
    \si{ns^{-1}}.}

\end{figure}

\begin{figure}
    
    \widecenter{\includempl{figctfitlaser-0}}
    
    \caption{\label{fig:ctfitlaser0} Fit of the DiCT models on laser pulses
    (Equations~\ref{eq:genpoisson} and~\ref{eq:geompoisson}). Left panels with
    overflow bin, right panels without. \scriptlink{figctfitlaser.py}}

\end{figure}

\begin{figure}
    
    \widecenter{\includempl{figctfitlaser-1}}
    
    \caption{\label{fig:ctfitlaser1} Fit of the DiCT models on laser pulses
    (Equations~\ref{eq:genpoisson} and~\ref{eq:geompoisson}), ``fixing'' the
    0~PE bin. Left panels with overflow bin, right panels without.
    \scriptlink{figctfitlaser.py}}

\end{figure}

\begin{figure}
    
    \widecenter{\includempl{figctfitap}}

    \figcaption{ctfitap}{Fit of the DiCT models on afterpulses
    (Equations~\ref{eq:borel} and~\ref{eq:geometric}). Left panels with
    overflow bin, right panels without.}
    
\end{figure}
