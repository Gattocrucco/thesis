\chapter{Analysis of correlated SiPM noise}
\label{ch:anal}

%     Spiego che i SiPM hanno del rumore e che lo studiamo
%     Gli obiettivi sono sia per il VETO che per la simulazione

In the previous chapters we studied the sources of noise in the procedure to
identify and measure a single signal. However the photodetector itself produces
pulses which are not due to a detected photon. We considered the following
sources of spurious signals:
%
\begin{description}
    
    \item[Dark count] The SiPM produces random pulses due to thermal
    excitations in the silicon.
    
    \item[Direct cross talk (DiCT)] The SiPM is composed of many single
    photodiodes, called \emph{cells}. When a cell fires, it emits thermal
    photons that can trigger nearby cells. The output is thus a single pulse
    but with an increased height proportional to the number of cells involved.
    
    \item[Afterpulses] The discharge can leave around electrons in metastable
    states that can more easily be excited by thermal fluctuations, thus after
    a pulse there is a probability of having a close successive pulse, with the
    probability decaying with delay from the originating pulse.

\end{description}

\marginpar{The DCR is both due to thermal excitations and tunneling. At 77 K
the tunneling should be prevalent (Savarese p.~59). There's the delayed cross
talk which I have not seen, it's mentioned in Savarese and Nagy 2014. Delayed
afterpulses too, but if the delayed cross talk is rare I guess delayed AP too.}

The DiCT and the afterpulses are called \emph{correlated noises} because they
are produced by and appear close to an initial source.

\section{Theory}
%         Un grafico con lo schema dei cross talk disegnato sulle celle
%         Modello per il DiCT
%         Decadimento degli afterpulse, spiegare la questione di quanti tau, l'altezza e la probabilità a delay piccoli
%         I numeri che ci aspettiamo circa di vedere (da Savarese)

\section{Data}
%         Elenco dei file con il percorso
%         Istogrammi 2D di tutti i file _2 (salto _1 perché a 5 VoV ha i doppi picchi), ricordarsi di togliere il digit 0 per non occupare il range dinamico con la saturazione
%         Controllo i trigger degli altri file
%         Doppi picchi

\section{Peak finding}

\subsection{Filtering}

\subsection{Baseline}

\subsection{Laser peak}

\subsection{Other peaks}

\subsection{PE}

\subsection{Filter deconvolution}

\section{Random pulses rate}

\section{Afterpulses}
%         Spiegare per bene come faccio i fit bayesiani con minimi quadrati e cosa significa statisticamente correggere con sqrt(chi2/dof)

\section{Direct cross talk}

\subsection{Random pulses}

\subsection{Afterpulses}

\subsection{Laser pulses}

\subsection{Results}

\section{Conclusions}

% VETO
% simulazione
% peak finding (citare LUX, dove stava? sta nei documenti che mi ha mandato Stracka?)
