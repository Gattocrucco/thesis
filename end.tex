\chapter{Conclusions}
\label{ch:end}

\marginpar{The ARMA filter. It is still not clear to me what they mean with it.
I would think a template for the peak and an AR for the tail, run on the
temporally reversed waveform because otherwise the exponential grows in the
wrong direction. This should reproduce the cross correlation filter since the
tail shape is well described by an exponential, as shown in \cite{luzzi2020}.
However, in \nolinkurl{results_on_MB2_tiles.pdf}, the performance is
substantially worse, and from the explanation they seem to not be reversing the
waveform. I asked Alessandro in a meeting is the filter was <<reversed\dots>>,
but he may have missed what I meant. Simone coherently said that it would
require buffering, so maybe Simone thinks it uses reversing? Maybe they changed
idea since 2018, realizing afterwards the possibility of reversal? Anyway, the
relevance here is saying that it may require too many resources due to the
reversal, otherwise one would think that it solves the problem completely.}

The main takeaway from \autoref{ch:snr} is in \autoref{fig:changebs}. The
maximum signal to noise ratio (SNR) that can be reached with filtering on that
data is~20, shown on the left panel with the cross correlation filter and
precise baseline calculation. However the constraints of the available
resources on the digitizers may turn out to impose a moving average filter with
imprecise baseline, lowering the SNR to~13, as shown in the central panel. So
the SNR at the digitizer stage can be as low as \SI{65}\% of the optimal one.

In \autoref{ch:timeres} there are various things to learn:

\begin{itemize}
    
    \item Upsampling is not necessary, \autoref{fig:rescomp}.
    
    \item The temporal resolution diverges below a certain SNR, giving a
    somewhat rigid bound on the working point, and the behavior at low SNR
    heavily depends on the noise spectrum, \autoref{fig:rescomp}.
    \texttt{Pyreco} currently simulates only white noise.
    
    \item For what concerns temporal resolution, it is sufficient to extract
    \SI{1}{\micro s} of waveform per pulse from the digitizers,
    \autoref{fig:windowtempres}. This is already shorter than what would be
    required for other reasons.
    
    \item Downsampling from \SI{125}{MSa/s} to \SI{62.5}{MSa/s} does not
    deteriorate the temporal resolution, \autoref{fig:tempresdowns} and
    \autoref{tab:filtsnrdowns}. If a compromise with the data rate is
    necessary, \SI{31.2}{MSa/s} may be still good enough.
    
\end{itemize}

\marginpar{Please, good sir, a reference for Pyreco!!}

Chapter~\ref{ch:rate} has its own summary in \autoref{sec:rateconcl}. In
general the formula for the fake rate is sufficiently precise, but there is an
exception with higher rate than expected that should be investigated better,
see \autoref{fig:fakerate}. With the tested filter, which matches the one
mentioned above for the $\mathrm{SNR} = 13$ figure, the fake rate is
\SI{10}{cps} at approximately 5 filtered noise standard deviations.

\marginpar{What is the maximum fake rate we can allow? Why? References?}

The chapter on correlated noise contains more material, so it includes a rather
long discussion of the results in \autoref{sec:analconcl}. To summarize
further:

\begin{itemize}
    
    \item Compared to the Fondazione Bruno Kessler (FBK) Tiles, the LFoundry
    Tiles have similar direct cross talk (DiCT) and much less afterpulsing (AP).
    
    \item The afterpulse models should be studied and validated better.
    
    \item It should be determined whether Tile~21 enters a nonlinear regime
    between \SI{7.5}{VoV} and \SI{9.5}{VoV}, or if the \SI{9.5}{VoV} data is
    mislabeled.
    
    \item Tile~21 should be analyzed with charge spectrum methods and the
    results compared to ours.
    
    \item The SiPM simulation must be consistent with the model used in the
    analysis of real data.
    
\end{itemize}

\marginpar{Here I would like to talk about the channel summing discussed in
the TF2 report, however I am not able to reproduce Simone's calculations which
give the $\SI{4.5}{VoV}\sqrt N$ formula. I have to have him explain the steps
to me. Link to the report: \url{https://www.overleaf.com/read/bcvnxwvvtnnf}.}
