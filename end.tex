\chapter{Conclusions}
\label{ch:end}

We studied the effect of electrical noise on the identification and
reconstruction of output pulses in DarkSide20k silicon photomultipliers (SiPM)
Tiles in \autoref{ch:snr}, \ref{ch:timeres} and~\ref{ch:rate}, and measured the
correlated noise of a Tile in \autoref{ch:anal}. In the following paragraphs we
summarize and comment the results from all chapters.

The main takeaway from \autoref{ch:snr} is in \autoref{fig:changebs}. The
maximum signal to noise ratio (SNR) that can be reached with filtering on that
data is~19, shown on the left panel with the cross correlation filter and
precise baseline calculation. However the constraints of the available
resources on the digitizers may turn out to impose a moving average filter with
imprecise baseline, lowering the SNR to~13, as shown in the central panel. So
the SNR at the digitizer stage can be as low as \SI{65}\% of the optimal one.

In the DarkSide collaboration it was proposed to use an ``autoregressive moving
average'' (ARMA) filter to implement a long cross correlation filter in the
digitizers. The ARMA combines linearly two filters: 1)~a cross correlation with
a template only for the peak of the pulse, and 2)~an autoregressive filter, run
on the \emph{temporally reversed} waveform. This reproduces a cross correlation
with the peak template plus an exponentially decaying tail. Although the
asymptotical complexity of the algorithm is excellent on paper, the actual
amount of logic resources necessary for the implementation on FPGA may exceed
the available ones, mainly due to the necessity of keeping a buffer for the
temporal reversal, and to glue appropriately the reversed temporal slices with
dynamical boundary conditions or partially overlapping slices. From
\autoref{fig:changebs} it is clear that a simple moving average can achieve
performances comparable to a cross correlation filter, thus it may not be
necessary to devise a way to implement the cross correlation filter on the
digitizers.

In \autoref{ch:timeres} there are various things to learn:

\begin{itemize}
    
    \item The temporal resolution diverges below a certain SNR, giving a
    somewhat rigid bound on the working point, and the behavior at low SNR
    heavily depends on the noise spectrum, \autoref{fig:rescomp}. The
    DarkSide20k simulation currently implements only white noise.
    
    \item For what concerns temporal resolution, it is sufficient to extract
    \SI{1}{\micro s} of waveform per pulse from the digitizers,
    \autoref{fig:windowtempres}.
    
    \item Downsampling from \SI{125}{MSa/s} to \SI{62.5}{MSa/s} does not
    deteriorate the temporal resolution, \autoref{fig:tempresdowns} and
    \autoref{tab:filtsnrdowns}. If a compromise with the data rate is
    necessary, \SI{31.2}{MSa/s} might still be good enough.
    
    \item Upsampling is not necessary, \autoref{fig:rescomp}.
    
\end{itemize}

The temporal resolution required by the specifications of DarkSide20k is
\SI{10}{ns} \cite[30]{aalseth2018}. A worse resolution would decrease the
selective power of pulse shape discrimination (PSD). With the noise spectrum of
the Proto0 prototype, we find that the temporal resolution reaches \SI{10}{ns}
at pre-filter SNR 2.6, to be compared with the SNR observed in the Proto0 data
of at least 3.3, see \autoref{fig:rescomp}.

Chapter~\ref{ch:rate} has its own summary in \autoref{sec:rateconcl}. In
general the formula for the fake rate is sufficiently precise, but there is an
exception with higher rate than expected that should be investigated better,
see \autoref{fig:fakerate}. With the tested filter, which matches the one
mentioned above for the $\mathrm{SNR} = 13$ figure, the fake rate is
\SI{10}{cps} with the threshold set at 5 filtered noise standard deviations
(including baseline subtraction), to be compared to the maximum total primary
noise rate required, \SI{250}{cps/PDM} \cite[30]{aalseth2018}.

The chapter on correlated noise contains more material, so it includes a rather
long discussion of the results in \autoref{sec:analconcl}. To summarize
further:

\begin{itemize}
    
    \item Compared to the Fondazione Bruno Kessler (FBK) Tiles, the analyzed
    LFoundry Tile has similar direct cross talk (DiCT) and much less
    afterpulsing (AP).
    
    \item The afterpulse models should be studied and validated better.
    
    \item It should be determined whether Tile~21 enters a nonlinear regime
    between \SI{7.5}{VoV} and \SI{9.5}{VoV}, or if the \SI{9.5}{VoV} data is
    mislabeled.
    
    \item Tile~21 should be analyzed with charge spectrum methods and the
    results compared to ours.
    
    \item The SiPM simulation must be consistent with the model used in the
    analysis of real data.
    
\end{itemize}

We put upper bounds on the dark count rate (DCR), respectively \SI{50}{cps},
\SI{170}{cps}, and~\SI{120}{cps} at overvoltages \SI{5.5}V, \SI{7.5}V,
and~\SI{9.5}V, to be compared with the required upper limit \SI{250}{cps}. We
give upper bounds for AP probabilities of \SI{2.5}\%, \SI{3.5}\% and
\SI{6.5}\%, and estimate DiCT probabilities \SI{20}\%, \SI{30}\% and~\SI{50}\%.
These are the probabilities of said noises being generated by a \SI1{PE} pulse.
The DarkSide20k specifications require the sum of correlated noise
probabilities to be less than \SI{60}\% \cite[30]{aalseth2018} in order to
avoid performance degradation, mainly dynamic range reduction on ionization
signals, where there is a lot of pile-up. Furthermore, some safety margin
should be considered for the aging of the SiPMs, which is not well studied, in
case the correlated noise probability turned out to increase with time. Recall
that, for a correlated noise probability $p$, the ratio of total pulses to
primary pulses is $1/(1-p)$, which diverges quite fast as $p$ goes to~1.
\cite[8]{gola2019} mention that in practice the maximum manageable probability
is \SI{50}\%.

\marginpar{Explain the stuff from Simone in the TF2 report.}
