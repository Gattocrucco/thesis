\chapter{Conclusions}
\label{ch:end}

The main takeaway from \autoref{ch:snr} is in \autoref{fig:changebs}. The
maximum signal to noise ratio (SNR) that can be reached with filtering on that
data is~20, shown on the left panel with the cross correlation filter and
precise baseline calculation. However the constraints of the available
resources on the digitizers may turn out to impose a moving average filter with
imprecise baseline, lowering the SNR to~13, as shown in the central panel. So
the SNR at the digitizer stage can be as low as \SI{65}\% of the optimal one.

\marginpar{Say that the ARMA filter may not fit into the digitizers, or would
complicate the design. Also, that the moving average is not so bad compared to
the cross correlation filter.}

In \autoref{ch:timeres} there are various things to learn:

\begin{itemize}
    
    \item Upsampling is not necessary, \autoref{fig:rescomp}.
    
    \item The temporal resolution diverges below a certain SNR, giving a
    somewhat rigid bound on the working point, and the behavior at low SNR
    heavily depends on the noise spectrum, \autoref{fig:rescomp}. The
    DarkSide20k simulation currently implements only white noise.
    
    \item For what concerns temporal resolution, it is sufficient to extract
    \SI{1}{\micro s} of waveform per pulse from the digitizers,
    \autoref{fig:windowtempres}. This is already shorter than what would be
    required for other reasons.
    
    \item Downsampling from \SI{125}{MSa/s} to \SI{62.5}{MSa/s} does not
    deteriorate the temporal resolution, \autoref{fig:tempresdowns} and
    \autoref{tab:filtsnrdowns}. If a compromise with the data rate is
    necessary, \SI{31.2}{MSa/s} may be still good enough, and we have not
    investigated whether in the latter case upsampling would recover some
    performance.
    
\end{itemize}

\marginpar{Add that the temporal resolution needed is \SI{10}{ns} from the
Yellow Book and say at which SNR we get there with Proto0 noise (2.6). Say that
Proto0 is considered a noisy environment, so it may be considered an upper
bound. Point to our definitions of SNR and temporal resolution.}

Chapter~\ref{ch:rate} has its own summary in \autoref{sec:rateconcl}. In
general the formula for the fake rate is sufficiently precise, but there is an
exception with higher rate than expected that should be investigated better,
see \autoref{fig:fakerate}. With the tested filter, which matches the one
mentioned above for the $\mathrm{SNR} = 13$ figure, the fake rate is
\SI{10}{cps} with the threshold set at 5 filtered noise standard deviations
(including baseline subtraction).

\marginpar{Take the maximum fake rate from the Yellow Book. The important is
total fake rate + dcr, but at this level it can be more since that figure is
after the FEP.}

The chapter on correlated noise contains more material, so it includes a rather
long discussion of the results in \autoref{sec:analconcl}. To summarize
further:

\begin{itemize}
    
    \item Compared to the Fondazione Bruno Kessler (FBK) Tiles, the LFoundry
    Tiles have similar direct cross talk (DiCT) and much less afterpulsing (AP).
    
    \item The afterpulse models should be studied and validated better.
    
    \item It should be determined whether Tile~21 enters a nonlinear regime
    between \SI{7.5}{VoV} and \SI{9.5}{VoV}, or if the \SI{9.5}{VoV} data is
    mislabeled.
    
    \item Tile~21 should be analyzed with charge spectrum methods and the
    results compared to ours.
    
    \item The SiPM simulation must be consistent with the model used in the
    analysis of real data.
    
\end{itemize}

\marginpar{Explain the stuff from Simone in the TF2 report.}
