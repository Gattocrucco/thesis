\chapter{Signal to noise ratio of filtered photodetection signals}
\label{ch:snr}

When a silicon photomultiplier (SiPM) detects a photon, the electrical output
is a sudden current spike, with a rise time on the order of nanoseconds, which
decays slowly in some microseconds.

This signal is immersed in electrical noise, which is proportional to the
square root of the area of the SiPM, since all photodiode cells are connected
in parallel. Given the relatively large area of the SiPMs Tiles, \SI{25}{cm^2},
the amplitude of the noise can be comparable to the amplitude of the signals.

This implies that to read out the detector it is important to filter the
waveform to suppress the noise. In this chapter we will study the performance
of some common filters on a PDM output.

\section{Data}
\label{sec:snrdata}

\begin{figure}[p]
    
    \widecenter{\includempl{fighist2dtile57-0}}
    
    \widecenter{\includempl{fighist2dtile57-1}}
    
    \figcaption{hist2dtile57}{Time-value histogram of Tile~57 in LNGS data.
    Top panel: whole events. Bottom panel: zoomed on the laser pulses and
    synchronized with the laser trigger.}
    
\end{figure}

\begin{figure}[p]
    
    \widecenter{\includempl{figsignals}}

    \figcaption{signals}{Single laser pulses.}

\end{figure}

We use the LNGS data for Tile~57, see \autoref{sec:lngsdata}. The data file is
\nolinkurl{MB2-LF-3x/NUV-LF_3x_57/nuvhd_lf_3x_tile57_77K_64V_6VoV_1.wav}. The
Tile is operated at \SI{6.0}{V} overvoltage, to be compared to \SI{6.5}{V} in
Proto0, see \autoref{fig:pdmadcch}. The amplitude of the signals is
proportional to the overvoltage, while the noise should not depend on it, so
this difference has to be taken into account when comparing the performances
in the Proto0 and LNGS setups.

In \autoref{fig:hist2dtile57} we show the persistence plot of the waveforms,
both on the whole acquisition window and zoomed on the pulses produced by the
laser. The persistence plot is a two-dimensional histogram of the events, where
the variables are the sample number in the event and the ADC value. It is like
stacking on each other all the waveforms. We have still not explained in full
detail how a SiPM works, for now we will do with what we can see with our own
eyes in the data. We defer an in-depth explanation to \autoref{sec:analtheory}.

From the whole event visualization it is evident that there are other pulses
occurring at random times beside those produced by the laser. We will need to
measure the height of pulses, and to do that we need a reference value, i.e., a
baseline. The straightforward way to compute the baseline is taking the average
of the pre-trigger region before the laser pulse, but this average would be
biased by the presence of a pulse. So we filter away all events where there is
a pre-trigger sample below~700, choosing this threshold by looking at
\autoref{fig:hist2dtile57}. There are~72 such events out of~\num{10005}.

Looking at the zoom on the laser pulses, we see that the amplitude of the
pulses is discrete. The laser does not emit only one photon, we expect a
Poisson distributed number of photons to hit the Tile. So we interpret the
discrete amplitudes as corresponding to the number of detected photons. We use
``PE'', standing for ``photoelectron'', as a unit of measure to indicate these
steps. So the \SI{1}{PE} pulses are the lowest ones, the first non-straight
band in the figure, then there are the \SI{2}{PE} pulses, and so on.

In \autoref{ch:anal} we will see that the PE do not correspond precisely to the
number of photons, due to the presence of other avalanche-generating
mechanisms, but for now we just care about the discretization. In
\autoref{fig:signals} we show some individual pulses with different PE.

\section{Filters}
\label{sec:filters}

A filter operates by converting the original sequence of ADC samples $(x_1,
x_2, \ldots)$ to a new ``filtered'' sequence $(y_1, y_2, \ldots)$. The filters
we consider are causal, i.e., the filtered sample $y_n$ can be computed only
using the original samples up to $x_n$. This limitation is introduced because
we are interested in using the filters online, updating the filter output
continuously as samples are read.

We tested three linear filters: the moving average, the exponential moving
average (also known as autoregressive filter), and the cross correlation
filter. These are standard filters and are the ones currently considered by the
DarkSide collaboration for the implementation of the online readout system.

We will always normalize the filters to behave like an average, in the sense
that a constant input waveform is mapped to itself. This is a convenient choice
because it allows to subtract the baseline value after filtering.

All filters have a temporal scale parameter, which we will tune in the analysis.

\subsection{Moving average}

The moving average consists in taking the average of the last $N$ samples:
%
\begin{equation}
    y_n = \frac1N \sum_{i=1}^N x_{n-N+i}.
\end{equation}

It is efficient to compute because at each step the average can be updated by
subtracting the $N$-th past sample and adding the incoming one:
%
\begin{equation}
    y_{n+1} = y_n + \frac1N x_{n+1} - \frac1N x_{n-N+1}.
\end{equation}
%
The memory requirement is a buffer with the $N$ samples.

Since the SiPM output is a current signal, the moving average is computing the
charge in a given time window, so it is also called ``charge integration'' in
this context.

\subsection{Exponential moving average}

The exponential moving average weighs past samples with an exponentially
decaying coefficient, and can be written recursively as
%
\begin{equation}
    y_n = a y_{n-1} + (1 - a) x_n, \quad a \in (0, 1).
\end{equation}

The scale of the exponential decay is given by
%
\begin{align}
    \tau &= -\frac1{\log a}, \\
    &\approx \frac1{1-a} \text{ for $a$ close to 1.}
\end{align}

This is the most efficient filter possible since it processes exactly once each
sample and requires no working memory.

\subsection{Cross correlation filter}

\begin{figure}
    
    \widecenter{\includempl{figautocorrlngs}}
    
    \figcaption{autocorrlngs}{Autocorrelation of the LNGS and Proto0 noise,
    computed on the part of the events before the trigger. The autocorrelation
    at lag $t$ is the correlation between a sample and another sample occurring
    a time $t$ after the first.}

\end{figure}

The cross correlation filter consists in computing the cross correlation of
the input with the expected signal shape. In some texts this is called matched
filter, term which we reserve for a different case explained later.

Let $\mathbf h = (h_1, h_2, \ldots, h_N)$ be a \emph{template} of the signal
waveform we want to detect. This means $\mathbf h$ should ideally match the
shape of the signal waveform we want to find in the noisy data. The filter is
then the cross correlation of $\mathbf x$ with $\mathbf h$:
%
\begin{equation}
    y_n = \sum_{i=1}^N h_i x_{n-N+i}, \quad \sum_{i=1}^N h_i = 1.
\end{equation}

Under the assumption that the data is white noise plus a signal that perfectly
matches the template apart from amplitude, this filter is optimal in the sense
that in the filter output there will be a peak corresponding to the signal and
this peak will have the maximum possible height relative to the standard
deviation of the filtered noise.

The differences we have from the ideal case are:
%
\begin{enumerate}
    \item the shape of the signal probably changes a bit each time;
    \item the signal is not aligned always in the same way to the ADC clock;
    \item most importantly, the noise is not white.
\end{enumerate}

We defer to \autoref{sec:cctemplate} the calculation of the template $\mathbf
h$. See \autoref{fig:filters} for an example of filtered waveform.

\begin{figure}
    
    \widecenter{\includempl{figfilters}}
    
    \figcaption{filters}{A filtered event. The number of averaged samples in
    the moving average filter, the scale of the autoregressive filter, and the
    length of the template of the cross correlation filter are all 1024
    samples, at \SI1{GSa/s}.}

\end{figure}

\subsection{Matched filter}

With the cross correlation filter, a non-white noise spectrum can be kept into
account by appropriately transforming $\mathbf h$, in this case the filter is
called \emph{matched filter} \cite[161]{ferrante2015}. Let $\mathbf s$ and
$\mathbf w$ be the signal and noise, such that the waveform to be filtered is
$\mathbf x = \mathbf s + \mathbf w$. Let $V$ be the noise covariance matrix,
i.e., $V_{ij} = \operatorname{Cov}[w_i, w_j]$. Then the template for the
matched filter is
%
\begin{equation}
    \mathbf h_\text m = V^{-1} \mathbf h.
    \label{eq:matched}    
\end{equation}

This filter is in some sense equivalent to doing a linear least squares fit of
the data with one parameter which multiplies the model $\mathbf h$. This can be
seen by writing the cross correlation with the template, $\mathbf h^\top V^{-1}
\mathbf x$, which compared to the least squares estimator \cite[628]{zyla2020}
is missing the multiplicative factor $1/(\mathbf h^\top V^{-1} \mathbf h)$.

We tried implementing the matched filter with mixed results, and we will not
report these studies in detail. We computed the noise covariance matrix on the
event samples before the signal. Since the noise is stationary, $V$ is a
Toeplitz matrix, i.e., the covariance depends only on the lag between two
samples. In \autoref{fig:autocorrlngs} we show the correlation of LNGS and
Proto0 noise.

The matched filter worked slightly better than the cross correlation filter, as
expected, but only for filter length $N$ sufficiently small, getting worse as
$N$ increased. This could be due to numerical accuracy problems in solving the
linear system $V$ in \autoref{eq:matched} as the size of the matrix increases.

\stracka{As we will see, the study on the simpler filters suggests possible
strategies to mitigate the numerical accuracy problems. \upshape Eh?}

The template obtained from the transformation is much different than the
initial signal shape. There are large cancellations, so it is more sensitive to
numerical accuracy, and may not be appropriate for an implementation on FPGA.
Moreover, it has not been considered by the rest of the collaboration, so we
did not investigate it further, and limited ourselves to the characterization
of the proposed filters.

\section{The fingerplot}
\label{sec:fingerplot}

To measure the performance of filters, we define a signal to noise ratio (SNR)
variable as follows: the SNR is the ratio of the value that, on average, the
peak of the filtered waveform yields on \SI1{PE} pulses (without noise), over
the standard deviation of the filtered noise.

We could consider other similar measures, for example we could include the
standard deviation of the peak filter output value since that influences where
we should place a threshold to discriminate signals, however it is sufficient
to use any reasonable definition for the purpose of comparing different
filters. Occasionally we computed the SNR without strictly adhering to the
definition above, and this will be signaled in the text where appropriate.

For each event we compute the filter output at a fixed temporal delay from the
leading edge of the laser trigger pulse, see \autoref{fig:filtersample}.
Then we compute the average of the samples before the trigger pulse and take
that value as ``baseline''. We subtract the baseline from the filter output,
and finally we change the sign to obtain positive values since the signals are
negative.

\begin{figure}
    
    \widecenter{\includempl{figfiltersample}}
    
    \figcaption{filtersample}{Example of how a signal amplitude is computed in
    an event for the purpose of computing the SNR. The samples in the shaded
    region are averaged to compute the baseline. The filter is evaluated at a
    fixed offset, indicated by the arrow, from the leading trigger edge. The
    amplitude then is the difference between the baseline and the filter value.
    The shaded region ends 100 samples before the trigger.}

\end{figure}

We take the list of these baseline and sign corrected filter outputs and
compute an histogram. One of these histograms is shown in
\autoref{fig:fingerplot}. It is called ``fingerplot'' due to the descending
peaks reminding of fingers.

\begin{figure}
    
    \widecenter{\includempl{figfingerplot}}
    
    \figcaption{fingerplot}{The fingerplot for the moving average filter with
    128 samples evaluated 128 samples after the trigger.}

\end{figure}

The first peak is centered in zero and thus corresponds to cases where no laser
photon is detected. Since the instant where we are evaluating the filter output
in each event is independent of the output itself, instead of, e.g., being
determined by peak finding, this peak is the distribution of the noise after
passing through the filter.

The various other peaks correspond to an increasing number of PE. The second
peak is the distribution of the filter output at a fixed instant for \SI1{PE}
signals. If the instant where we evaluate the filter is the one associated to
the highest signal response, the mean of the second peak divided by the
standard deviation of the first yields the SNR.

Since the peaks are often overlapping, and that we will have to repeat the
calculations for many fingerplots automatically without checking each one, we
use robust measures of location and scale instead of the average and standard
deviation. We run a peak finding algorithm on the fingerplot and divide the
data by putting boundaries halfway between peaks, so that we assign each
datapoint to a peak. For the second peak, we take the median instead of the
average. For the first peak, we take half of the symmetric \SI{68}\%
interquantile range instead of the standard deviation, i.e., half the
difference between the 0.84 and 0.16 quantiles.

On a Gaussian distribution these are equivalent to the mean and standard
deviation, however they are less sensible to the tails of the distribution,
which is a desirable property since the boundaries could cut away the tails and
there could be contamination from the tails of the neighboring peaks. By
definition, when the SNR is high, the peaks are well separated, so in that case
the result can be trusted to match precisely the quantiles of the single
isolated peak.

Moreover, the code always automatically checks that the median of the \SI0{PE}
peak is zero within its error. If it is not, the SNR is considered to be zero,
such that it is well visible in the results.

\section{Cross correlation filter template}
\label{sec:cctemplate}

We compute the template for the cross correlation filter by taking the average
of \SI1{PE} signals, selected using the fingerplot boundaries obtained with a
1500-samples average, counting the samples starting from the trigger position.
The baseline is computed with the median. We consider three alignment methods
for the waveforms prior to averaging:
%
\begin{itemize}
    
    \item unaligned, or rather aligned to the acquisition window;
    
    \item aligned using the laser trigger;
    
    \item aligned using a cross correlation filter done with the unaligned
    template, where the alignment offset is the position of the minimum in the
    filter output.

\end{itemize}

We show the obtained templates in \autoref{fig:templates}. The trigger square
pulse is sampled at \SI1{GSa/s} like the PDM output, so we assume that it gives
a precise alignment. Indeed, the pulse peak appears sharper than with the
unaligned template. The filter-aligned template has an additional sinc-like
oscillation centered on the peak. It is similar to the autocorrelation of the
noise, see \autoref{fig:autocorrlngs}, so we interpret it as the effect of a
non-null coherence in the averaged noise, due to the fact that the position of
the filter peak minimum will depend both on the position of the signal and on
the overlapping noise oscillations, which have a width similar to the one of
the signal peak.

\begin{figure}
    
    \widecenter{\includempl{figtemplate}}
    
    \figcaption{template}{Templates obtained averaging \SI1{PE} laser pulses,
    selected with a \SI{1.5}{\micro s} charge integration fingerplot, with
    different pre-averaging alignment choices. The template we pick is the
    trigger-aligned one.}

\end{figure}

The trigger-aligned template appears to be the best one, so we will use it in
this chapter and in \autoref{ch:timeres}. However \autoref{ch:anal} analyzes
data without trigger information, forcing us to use the filter-aligned template
in that case. See \cite{coakley2001} for an alignment method that may offer
some improvements over our vanilla approach.

When using the template, we normalize it to unit sum, as explained in
\autoref{sec:filters}. Since we will try various lengths of the filter
template, we have to decide how to truncate the full template. When truncating
to $N$ samples, we pick $N$ contiguous samples from the template such that
their sum of squares is maximized. Of course normalizing the template to unit
sum is done after truncation.

We did not smooth the template nor fitted it with a model. For how to fit this
kind of pulses, we refer to, e.g., \cite{luzzi2020}. We think that our vanilla
average is fine enough, but we did not actually compare it with a fitted
template.

We conclude this section with a technical detail: when computing the
filter-aligned template, we first smooth the unaligned template with a 2-sample
moving average. This is to remove a coherent noise at the Nyquist frequency,
i.e., even samples are always biased in the same direction relative to odd
samples, in all waveforms. If we do not correct this, the noise appears in the
template, then when using it the filter matches the noise in template to the
one in the waveform, and gives a peak output only on even or odd samples.

\section{SNR versus filter length}

When determining how to compute the SNR from the fingerplot we assumed that we
were evaluating the filter output at the optimal instant. Since we do not know
it a priori, we repeat the computation for a range of values of the filter
evaluation point. A simpler solution that comes to mind is taking the minimum
(the signals are negative) of the filter output in each event, however this
would bias upward the SNR measure because the minimum can yield lower values
due to noise peaks. Instead, by fixing the point independently from the data,
the random oscillation due to noise is symmetrical and the averaging recovers
the actual amplitude of the filter output for the signal.

The resulting SNR curves are shown in \autoref{fig:snrplot}, repeated for a
range of values of the filter-length parameter. For the moving average and
cross correlation filter, the length parameter is the number of samples, $N$.
For the exponential moving average, it is the scale of the exponential decay
$\tau$.

\stracka{Specifica la baseline usata per questo plot. \upshape Risposta:
viene dal fingerplot, spiegato nel suo capitolo.}

\begin{figure}
    
    \widecenter{\includempl{figsnrplot}}
    
    \figcaption{snrplot}{The SNR as a function of delay from trigger (x-axis)
    and filter length (shade of gray) for the three filters.}

\end{figure}

The maximum of each curve gives the actual SNR figure we are interested in. We
expect by intuition that the width of the maximum is approximately proportional
to the temporal resolution we could achieve if we used the filter output to
locate temporally the signal. So we do another plot, \autoref{fig:changebs}
left column, where we show the maximum SNR value and the width of the peak
versus the filter length parameter. We measure the width as the distance
between the two points where the SNR is 1 less than its maximum value.

\section{Effect of the baseline computation}

In constructing the fingerplot, we subtract from the filter output value the
baseline, i.e., the average value of the waveform in absence of signals. We
compute the baseline for each event as the average of the samples before the
signal. As a default, we average \num{8000} samples. The value obtained varies
randomly due to the noise; its standard deviation is that of the noise divided
by $\sqrt{8000}$. The maximum filter length we use is 2000, so in that case the
width of the peaks gets an additional contribution (summed in quadrature) of
roughly $\sqrt{2000/8000} = 1/2$ of the width we would have with a noiseless
baseline, i.e., the width of the first peak is $\sqrt{1 + 1/2} - 1 = \SI{22}\%$
larger, lowering the SNR by the same percentage. We note that, since the noise
is autocorrelated, the standard deviation of the mean does not scale with
$1/\sqrt N$, giving a more detrimental effect on the SNR than the one just
estimated, and the width of the \SI0{PE} peak depends on the filter.

These considerations are relevant to real-world applications, because the
baseline computation is a relevant part of the signal finding, and it requires
either a fast on-digitizer algorithm estimating it reliably or sending enough
samples to subsequent stages in the data processing chain. The value of the
baseline is expected to change in the detectors, so it may not be feasible to
compute it once with a very long average and then keep it fixed for a while.

To get an idea of the effect of the precision of the baseline calculation, we
repeat everything computing the baseline with just 1000 or 200 samples. The
result is in the central and rightmost columns of \autoref{fig:changebs}. We
see that, for example, going from 8000 to 200 baseline samples the maximum SNR
drops from 19 to~9.

It may appear strange that, with an imprecise baseline, the SNR starts dropping
with increasing filter length at somewhat short lengths. We note that the
filters behave like an average, so their peak value on the signal decreases as
they get longer. The filtered noise contribution decreases too, but the
baseline error is constant.

\begin{figure}[t]
    
    \widecenter{\includempl{figchangebs}}
    
    \figcaption{changebs}{Top row: maximum SNR achieved with each filter
    (different curves) and filter length (x-axis). Bottom row: the width of the
    maxima, computed as the length of the interval of offset from trigger with
    endpoints where the SNR is 1 less than the maximum. The cross correlation
    filter maintains almost constant width with increasing SNR, while the other
    two filters have increasing width. In each column the computation is
    repeated changing the number of pre-signal samples used to compute the
    baseline. The performance decreases with a lower number of samples because
    the error on the baseline increases and the baseline is compared to the
    filtered signal amplitude to discriminate signals from noise.}

\end{figure}

\section{Noise spectrum}
\label{sec:spectrum}

To support our previous statement that the noise is not white, in this section
we report the frequency spectrum of the noise.

We take the region of each event before the trigger pulse, compute its
periodogram, and take the median across events for each frequency. We use the
median instead of the average in case there were spurious signals or
irregularities we missed. We do the same for the same tile (57) in Proto0 data,
with no need to isolate the pre-trigger region in that case. The spectra thus
obtained are shown in \autoref{fig:spectra}. We also compute the spectrum of
the signal template and compare it to the noise in \autoref{fig:templsp}. Note
that we do not window the signal because it is not a stationary process.

\begin{figure}[t]
    
    \widecenter{\includempl{figspectra}}
    
    \figcaption{spectra}{Noise spectrum of Tile~57 in LNGS and Proto0. The
    vertical scale has arbitrary units.}

\end{figure}

\begin{figure}[b]
    
    \widecenter{\includempl{figtemplsp}}
    
    \figcaption{templsp}{Comparison of the spectra of signal and noise. The red
    curve shows the percentage of signal power below a given frequency. The
    signal spectrum is computed with the discrete Fourier transform of the
    cross correlation filter template (\autoref{fig:template}) without
    windowing.}

\end{figure}
