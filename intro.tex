\chapter*{Introduction}

The goal of this thesis is to do some studies on the silicon photomultipliers
(SiPMs) photodetection modules (PDMs) that will be used in the DarkSide20k dark
matter detection experiment, needed in order to decide some details of the data
acquisition system (DAQ) and of the layout of the readout electronics. Similar
work in the DarkSide collaboration has been reported in
\cite[ch.~3,~5]{savarese2018}.

\marginpar{Officially, is it photo\emph{detection} or \emph{detector} modules?}

The thesis is divided in eight chapters. The first two chapters provide a brief
introduction to dark matter and to the DarkSide experiment. They do not contain
new material and are included only for the sake of completeness.
Chapter~\ref{ch:data} is a reference for the sources of the datasets used in
the analyses. Chapters~\ref{ch:snr}, \ref{ch:timeres} and~\ref{ch:rate} deal
with the performance of single pulse detection in the PDM output (signal to
noise ratio, temporal resolution, and fake rate). Chapter~\ref{ch:anal}
analyzes additional pulses noise (cross talk and afterpulsing).
Chapter~\ref{ch:end} draws the conclusions.

This layout has more chapters than is currently conventional for this kind of
document. This is done to keep each somewhat self-contained topic well
separated, such that it should be quicker to pull out specific information.

To the same end, for each figure and table we provide a Python script that
reproduces the content. Each script is referenced at the end of the caption
like this: \scriptlink{fignoise.py}. In the PDF it links is to a preview of the
file in an online repository, \url{https://github.com/Gattocrucco/sipmfilter}.
To run these scripts, clone or download the repository. The scripts are located
in the directory \nolinkurl{figthesis}. The file \nolinkurl{README.md} provides
detailed instructions on how to set up the working environment.

Some scripts require no input. Others require large data files which are not
included in the repository. Of the latter, some have a cache of the data they
need, others just can not be executed without the original data. For people
outside of the DarkSide collaboration it may not be possible to obtain the data
files; in any case, they can check what the code is doing exactly.

The \LaTeX{} code for this thesis is itself available at
\url{https://github.com/Gattocrucco/thesis}. Moreover, the
\nolinkurl{sipmfilter} repository contains slides on this work which we
presented at DarkSide meetings, although they shall be considered outdated
respect to this document. The Python code is covered by the MIT license, while
the thesis and the slides are covered by the CC-BY~4.0. These licenses grant
anyone the right to reuse this material, even without releasing themselves the
modifications as open source, provided they cite the original author and keep
the copyright and license notices. The licenses do not cover figures which we
copied from other sources nor the DarkSide data.
