\chapter*{Introduction}

DarkSide20k is a planned dual-phase liquid argon (LAr) time projection chamber
(TPC) designed to detect dark matter, the successor to DarkSide-50. It will be
the largest detector of its kind, with 20~metric tons of argon in the fiducial
volume. The predicted resulting upper bound on the spin-independent
WIMP-nucleon scattering cross-section, in case of no discovery, is
$\approx\SI{1e-47}{cm^2}$ at \SI{1}{TeV/c^2} WIMP mass, to be compared with the
current best limit $\approx\SI{1e-45}{cm^2}$ by XENON1T.

DarkSide20k is part of a larger effort to improve the sensitivity to elastic
nuclear recoils down to the ultimate goalpost of coherent neutrino scattering.
The expected number of neutrino recoils in the \SI{100}{t yr} exposure of
DarkSide20k is approximately~1. It is thus worthwhile to pursue this effort
since we are indeed close to the conclusion. If a WIMP-like signal was not
found before then, new detection techniques would be needed to continue the
WIMP search, in particular, sensitivity to the recoil direction would permit to
push down further the limits.

In the predecessor DarkSide-50, \SI{90}\% of the neutron background was traced
to originate in the photomultiplier tubes (PMTs). To scale from the \SI{50}{kg}
of argon in DarkSide-50 to the 20~tons of DarkSide20k, it is then necessary to
replace the PMT with a more radio-pure photodetector. A key enabling technology
is thus the silicon photomultiplier (SiPM), which has better radio-purity. Of
the various advantages over the PMT, another important one is single-photon
resolution, needed because the isotropic scintillation signals will be very
dispersed in the large TPC of DarkSide20k, with often only one photon hitting a
photodetector.

The readout of SiPMs, however, requires more complicated processing than PMTs.
First, to equip its uniquely large photodetection area, DarkSide20k will employ
large \SI{25}{cm^2} photodetectors. A SiPM of this size produces an high amount
of floor electrical noise. Furthermore, the SiPM suffers from correlated noise,
i.e., secondary output pulses induced by other pulses instead of by incident
light or dark count. The generation of secondaries is recursive, causing loss
of dynamic range through pile-up and leading eventually to saturation if not
kept under control.

In this thesis we present reconstruction and characterization studies on the
photodetector modules (PDMs) that will be used in the TPC. These studies
quantify the effects and the amount of noise in the SiPMs and are primarily
meant as a support to the definition of the first stages of the online
processing chain.

Each PDM consists of a \SI{25}{cm^2} $6\times 4$ matrix of individual SiPMs,
and a front end board (FEB) that pre-amplifies and sums analogically the output
of the SiPMs. Thus qualitatively its behavior is akin to a single large SiPM.
The SiPM has Geiger-mode single photon response, i.e., each detected photon
produces one fixed amplitude pulse. The pulse looks like a sharp peak,
$\approx\SI{20}{ns}$, followed by a rather long exponential tail,
$\approx\SI{1}{\micro s}$. SiPMs have three kinds of noise: 1)~stationary
electric noise, which scales with the square root of the area; 2)~a dark count
rate (DCR) of pulses independent of incident light that scales with the area;
3)~the correlated noise produced by primary pulses, which contributes a factor
proportional to the DCR and photon pulses.

The first two stages in the readout chain will be the digitizers and the front
end processors (FEP). The digitizers find candidate pulses, and for each one
send a slice of waveform to the FEP, where the final identification of pulses
is decided. The performance of these stages is mainly determined by the
electric noise, characterized with the signal to noise ratio (SNR), which is
the ratio of the amplitude of pulses over the noise standard deviation. It
influences the fake rate, i.e., the rate of random oscillations high enough to
be mistakenly identified as pulses, and the temporal resolution of pulse
detection.

By applying linear filters to digitized waveforms acquired from the PDMs
illuminated by a pulsed laser, both in a testing setup at Laboratori Nazionali
del Gran Sasso (LNGS) and in the small prototype TPC ``Proto0'', we study the
noise parameters of single pulse detection: SNR, temporal resolution, fake rate.

We consider 1)~an autoregressive filter, which uses the least possible
computational resources, 2)~a matched filter without spectrum correction, which
gives almost optimal performance, 3)~a moving average, which is a compromise
between simplicity and performance. Simple filters are needed on the
digitizers, which must process all the incoming data, while the FEP will
probably use the optimal filter. We also study the baseline computation and the
filter length.

Then using a custom peak finder algorithm we measure the DCR and study the
correlated noise, which consists of additional pulses produced recursively by
each pulse, divided in two main categories: afterpulses (AP), which arrive with
some delay from the parent pulse, and have smaller amplitude as the delay goes
to zero, and direct cross talk (DiCT), which manifests as an integer
multiplication of the amplitude of pulses because the children pulses are
overlapped with the parent.

The thesis is divided in eight chapters. The first two chapters provide a short
pedagogical introduction to dark matter and to the DarkSide experiment.
Chapter~\ref{ch:data} is a reference for the sources of the datasets used in
the analyses. Chapters~\ref{ch:snr}, \ref{ch:timeres} and~\ref{ch:rate} deal
with the performance of single pulse detection in the PDM output: signal to
noise ratio, temporal resolution, and fake rate. Chapter~\ref{ch:anal} analyzes
correlated noise. Finally, \autoref{ch:end} summarizes the key results from
each chapter. The computer code implemented for this thesis is released as
open-source, see \autoref{ch:code}.
