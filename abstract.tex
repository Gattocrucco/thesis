\documentclass[11pt]{article}

\usepackage{geometry}
\usepackage[T1]{fontenc}
\usepackage{siunitx}
\usepackage{lineno} % \linenumbers
\usepackage{fancyhdr}

\newlength\bindingoffset
\setlength\bindingoffset{1cm}
\geometry{%
    a4paper,
    asymmetric, % twoside, but marginpar are always to the right
    centering,
    textwidth=360pt, % default LaTeX textwidth with 11pt (345pt for 10pt)
    top=1.8cm,
    bottom=1.8cm,
    marginparwidth=3.4cm,
    headsep=10pt,
    footskip=17pt,
    bindingoffset=\bindingoffset,
    % showframe,
}
\addtolength\marginparwidth{-0.5\bindingoffset}

\linenumbers % print line numbers (for the draft)

% small sans font for marginpar
\let\oldmarginpar\marginpar
\renewcommand\marginpar[1]{\oldmarginpar{\sffamily\scriptsize #1}}

\newlength\pagenumbermargin
\setlength\pagenumbermargin{2.9cm}
\addtolength\pagenumbermargin{-0.5\bindingoffset}

\newcommand\hdrside{RO,LE}
\newcommand\sharedstyle{%
    \renewcommand\headrulewidth{0pt}
    \fancyhf{}
    \fancyhfoffset\pagenumbermargin
    \fancyfoot[\hdrside]\thepage
}

\fancypagestyle{plain}\sharedstyle
\pagestyle{fancy}
\sharedstyle

\author{Giacomo Petrillo\\
Supervisors: Eugenio Paoloni, Simone Stracka\\
University of Pisa}

\title{Thesis abstract: Online processing of the large area SiPM detector
signals for the DarkSide20k experiment}

\begin{document}
    
    \maketitle

    The thesis consists in some studies on the silicon photomultipliers (SiPMs)
    photodetector modules (PDMs) that will be used in the DarkSide20k dark
    matter detection experiment, needed in order to decide some details of the
    data acquisition system (DAQ) and of the layout of the readout electronics.
    
    DarkSide20k is a planned dual-phase liquid argon (LAr) time projection
    chamber (TPC), the successor to DarkSide-50. It will be the largest
    detector of its kind, with 20~metric tons of argon in the fiducial volume.
    The predicted resulting upper bound on the spin-independent WIMP-nucleon
    scattering cross-section, in case of no discovery, is
    $\approx\SI{1e-47}{cm^2}$ at \SI{1}{TeV/c^2} WIMP mass, to be compared
    with the current best limit $\approx\SI{1e-45}{cm^2}$ by XENON1T.
    
    The photodetectors of the TPC will employ SiPMs instead of the usual
    photomultiplier tubes (PMTs). The SiPM has Geiger-mode single photon
    response, i.e.\ each detected photon produces one fixed amplitude pulse.
    The photodetection efficiency is higher than PMTs, reaching \SI{50}\%, with
    also a better geometric occupancy, which should reach \SI{80}\% in
    DarkSide20k.
    
    SiPMs have three kinds of noise: stationary electric noise, which scales
    with the square root of the area; a dark count rate (DCR) of pulses
    independent of incident light that scales with the area; and correlated
    noise produced by primary pulses, which contributes a factor proportional
    to the DCR and photon pulses.
    
    The pulse shape is a sharp peak followed by a long exponential tail. By
    applying linear filters to digitized waveforms acquired from the PDMs
    illuminated by a pulsed laser, both in a testing setup at Laboratori
    Nazionali del Gran Sasso (LNGS) and in the small prototype TPC Proto0, we
    study the noise parameters of single pulse detection: signal to noise ratio
    (SNR), temporal resolution, and fake rate. These are all related to the
    electrical noise.
    
    Then using a custom peak finder algorithm we measure the DCR and study the
    correlated noise, which consists in additional pulses produced recursively
    by each pulse, divided in two main categories: afterpulses (AP), which
    arrive with some delay from the parent pulse, and have smaller amplitude as
    the delay goes to zero, and direct cross talk (DiCT), which manifests as a
    integer multiplication of the amplitude of pulses because the children
    pulses are overlapped with the parent.
    
    The results are the following. The resources on the digitizers may limit
    the SNR after filtering to~13, compared to~20 which would be obtained with
    an almost optimal filter. After identifying a candidate pulse, the
    digitizers will send a slice of waveform to the front end processing (FEP).
    
    The temporal resolution matters on the FEP where the final identification
    of pulses is decided. For what concerns temporal resolution: 1)~upsampling
    is not necessary; 2)~at low SNR the resolution diverges and how fast
    heavily depends on the noise spectrum, with the Proto0 noise the maximum
    allowed by specs, \SI{10}{ns}, is reached at pre-filter SNR~2.6; 3)~it is
    sufficient to have \SI{1}{\micro s} of waveform per pulse; 4)~it is
    possible to lower the sampling frequency from \SI{125}{MSa/s} to
    \SI{62.5}{MSa/s}.
    
    Using a simple filter which surely fits into the digitizers, the fake rate
    is \SI{10}{cps} when the threshold is set to 5~standard deviations of the
    filtered noise, where the filter includes the subtraction of the baseline.
    Since the filter procedure is not actually decided yet, we give
    instructions on how to compute the fake rate without actually counting the
    threshold crossings with \SI{1}{ms} of recorded data. Of the 25~PDMs we
    looked at, one has an anomalously high fake rate which we did not
    investigate properly.
    
    We give upper bounds for the DCR of a $6\times4$ SiPMs Tile at three
    overvoltages, \SI{5.5}{V}, \SI{7.5}{V}, \SI{9.5}{V} (the overvoltage is the
    difference between the bias put on the SiPM and the breakdown voltage of
    the junction), which are respectively \SI{50}{cps}, \SI{170}{cps},
    and~\SI{120}{cps}, to be compared with the DarkSide20k requirement of
    \SI{250}{cps}. \SI{5.5}{V} is a somewhat usual operating overvoltage while
    \SI{9.5}{V} is considered high. Increasing the overvoltage increases both
    the SNR, the DCR and the correlated noise.
    
    The analysis of correlated noise, done on the same data, gives upper bounds
    for AP probabilities of \SI{2.5}\%, \SI{3.5}\% and \SI{6.5}\%, and DiCT
    probabilities \SI{20}\%, \SI{30}\% and~\SI{50}\%. These are the
    probabilities of said noises being generated by any given single pulse,
    i.e.\ the stacked pulses produced by DiCT count separately. The maximum
    DiCT probability tolerable during operation is reported to be \SI{50}\%.
    The DarkSide20k specifications require less than \SI{60}\% DiCT+AP to
    maintain a good dynamic range on ionization signals, where there is a lot
    of pile-up.
    
    Measuring AP and DiCT requires models. We try the models we find in the
    literature and in the DarkSide20k simulation and bring them into question,
    but we do not search for better alternatives. They are probably good enough
    for the necessities of the experiment. We find that the AP temporal
    distribution is well described by two exponential decays with constants
    \SI{200}{ns} and \SI{1}{\micro s}, but not by a single one.
    
    Finally as an appendix we give a Bayesian interpretation of the common
    procedures used to fit histograms with least squares.
    
\end{document}
